\documentclass[12pt,letterpaper]{article}
\usepackage{fullpage}
\usepackage[top=2cm, bottom=4.5cm, left=2.5cm, right=2.5cm]{geometry}
\usepackage{amsmath,amsthm,amsfonts,amssymb,amscd}
\usepackage{lastpage}
\usepackage{enumerate}
\usepackage{fancyhdr}
\usepackage{hyperref}
\usepackage{pythontex}
\usepackage[siunitx]{circuitikz}
\usepackage{xcolor}
\usepackage[inline]{enumitem}
\usepackage{booktabs}
\usepackage{relsize}
\usepackage{calculator}
\usepackage{siunitx}
\usepackage{tikz}

\setlength{\parindent}{0.0in}
\setlength{\parskip}{0.05in}

% Edit these as appropriate
\newcommand\course{MCEN 5228-004}
\newcommand\Homework{1}                  % <-- homework number
\newcommand\NetIDa{Coovi Meha}           % <-- NetID of person #1
\newcommand\NetIDb{MEid:481-473}           % <-- NetID of person #2 (Comment this line out for problem sets)

\pagestyle{fancyplain}
\headheight 35pt
\lhead{\NetIDa}
\lhead{\NetIDa\\\NetIDb}                 % <-- Comment this line out for problem sets (make sure you are person #1)
\chead{\textbf{\Large Homework \Homework}}
\rhead{\course \\ \today}
\lfoot{}
\cfoot{}
\rfoot{\small\thepage}
\headsep 1.5em
\begin{document}
\begin{pycode}
\end{pycode}
\section{Problem  Statement}
Find the synthesis equation \(C_k\) of the following fourier serie:\\
The sampling property of  \(\delta(t)\) is: \\
\begin{equation}
    \int_{-\infty}^{\infty} f(t)\delta(t-a) dt = f(a)
\end{equation}
\begin{figure}[h]
    \begin{tikzpicture}
        \draw  (0,0) -- (14, 0); 
        \draw [thick,->](1,0)node[anchor = north west]{\(-4T_{0}\)} -- (1,4.5);
        \draw [thick,->](3,0) node[anchor = north west]{\(-3T_{0}\)}-- (3,4.5);
        \draw [thick,->](5,0) node[anchor = north west]{\(-2T_{0}\)}-- (5,4.5);
        \draw [thick,->](7,0) node[anchor = north west]{\(T_0\)} -- (7,4.5);
        \draw [thick,->](9,0) node[anchor = north west]{\(2T_0\)}-- (9,4.5);
        \draw [thick,->](11,0) node[anchor = north west]{\(3T_0\)}-- (11,4.5);
        \draw [thick,->](13,0) node[anchor = north west]{\(4T_0\)}-- (13,4.5);
    \end{tikzpicture}
    \caption{Representation of \(\delta_{T_0}(t)\)}
\end{figure}
\section{Solution}
    \(f(t)= \sum_{-\infty}^{\infty}  A\delta(t-mT_0)\)\\
     \(f(t)=  A\delta(t)\) by replacing this into (1) we get the following equation
     \begin{equation}
        f(a)= \int_{-\infty}^{\infty}  A\delta(t)\delta(t-a) dt
    \end{equation}
         \(f(a)= \int_{-\infty}^{\infty}  A\delta(t)\delta(t-a)\) \\
    \begin{equation}
        C_k = \frac{1}{T_0} \int_{\frac{-T_0}{2}}^{\frac{T_0}{2}} f(t) e^{-jw_0t} 
    \end{equation}
 with \(e^{-jw_0t}=1\) we can conclude that \(C_k=\frac{1}{T_0}f(a)\)\\
 replace f(a) into this equatio  yeild\\
 \begin{equation}
     C_k= \frac{1}{T_0}\int_{-\infty}^{\infty}  A\delta(t)\delta(t-a) dt
 \end{equation}
 The integral is zero ouside the interval of \(-T_0\) and \(T_0\), thus
 \begin{equation}
    C_k= \frac{1}{T_0}\int_{\frac{-T_0}{2}}^{\frac{T_0}{2}}  A\delta(t)\delta(t-a) dt
\end{equation}
The area under any impulse signal is always zero. therefore we have 
\begin{equation}
    C_k= \frac{A}{T_0}
\end{equation}
\end{document}

